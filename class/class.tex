\documentclass[UTF8]{ctexart}
\usepackage{listings,xcolor,microtype,mathtools,amsmath,graphicx,array,fix-cm,newtxmath,indentfirst,geometry,latexsym}
\lstset{
	numbers=left, 
	numberstyle= \tiny, 
	keywordstyle= \color{ blue!70},
	commentstyle= \color{red!50!green!50!blue!50}, 
	frame=shadowbox, % 阴影效果
	rulesepcolor= \color{ red!20!green!20!blue!20} ,
	escapeinside=``, % 英文分号中可写入中文
	xleftmargin=2em,xrightmargin=2em, aboveskip=1em,
	framexleftmargin=2em
}
\geometry{a4paper,left=2cm,right=2cm,top=1cm,bottom=1cm}
\title{关于类的三大特性综述}
\date{}
\begin{document}
	\maketitle
\begin{flushleft}
\qquad 类是用来描述具有相同属性和方法的对象的集合。其主要有三大特性,即继承、多态、封装(私有化),下文将分别进行阐述。
\subsection*{1\;继承(Inheritance)}
\qquad 类的一个主要功能就是“继承”。“继承”一词从字面理解是对象A从对象B处获得某物。类似地,在面向对象编程语言里,是指一种可以使用现有类的所有属性、方法,无需重新编写,同时可在原来类的情况下,对某些方法属性进行扩展或重新定义。那么,通过继承创建的新类称为“子类”或“派生类”,被继承的类称为“基类”、“父类”或“超类”。例如:子类B若要继承父类A,则在B定义括号中需写父类名称A。\\

\begin{lstlisting}
class A:
    m=6
class B(A):          
    pass
\end{lstlisting}
{\textbf{1.1\:调用父类属性方法}}
\begin{lstlisting}
class Father():
    def _init_(self):
        self.a='aaa'  
    def action(self):
        print('`调用父类的方法`')
class Son(Father):
    pass

son=Son()        #`子类Son继承父类Father的所有属性和方法`
son.action()     #`调用父类方法`
son.a            #`调用父类方法`
\end{lstlisting}

{\textbf{1.2\:重写父类属性方法}}
\begin{lstlisting}
class Father():
    def _init_(self):
        self.a='aaa'

    def action(self):
        print('`调用父类的方法`')

class Son(Father):
    def _init_(self):
        self.a='bbb'
    def action(self):
    print('`重写父类的方法`')

son=Son()       #`子类Son继承父类Father的所有属性和方法`
son.action()    #`子类Son调用自身的action方法而不是父类的action方法`
son.a
\end{lstlisting}
{\textbf{1.3\:继承的分类}}\\
\qquad 继承按照继承方式,一般分为单继承与多继承。单继承即是只从一个父类那继承,而多继承则是指继承的父类不止一个,可通过多级继承实现。按照实现方式可分为实现继承、接口继承和可视继承三类。实现继承是指使用父类的属性和方法而无需额外编码的能力;接口继承是指仅使用属性和方法的名称,但子类须提供实现的能力;可视继承是指子类使用父类的外观和实现代码的能力。\\
\qquad 在考虑使用继承时,需注意两个类之间的关系应为“属于”关系,且在子类中不能定义和父类同名的实例变量。\\

\subsection*{2\;多态(Polymorphisn)}
\qquad 继承让类与类之间产生类关系,提供了多态的前提。在编程语言和类型论中,多态指为不同数据类型的实体提供统一的接口。多态性是指具有不同功能的函数可以使用相同的函数名,这样就可以用一个函数名调用不同内容的函数。在面向对象方法中一般是这样表述多态性:向不同的对象发送同一条消息,不同的对象在接收时会产生不同的行为(即方法)。也就是说,每个对象可以用自己的方式去响应共同的消息(消息,就是调用函数,不同的行为就是指不同的实现,即执行不同的函数)。相比Java,Python中也支持多态,但是是有限的的支持多态性,主要是因为python中变量的使用不用声明,所以不存在父类引用指向子类对象的多态体现。实现多态,通常有两种方式:覆盖,重载。覆盖,是指子类重新定义父类的虚函数的做法。重载,是指允许存在多个同名函数,而这些函数的参数表不同。在python中没有重载的概念。\\
\qquad Python中多态的作用主要是让具有不同功能的函数可以使用相同的函数名,这样就可以用一个函数名调用不同内容(功能)的函数。同时,Python中多态的特点主要有以下几点:
(1)只关心对象的实例方法是否同名,不关心对象所属的类型;
(2)对象所属的类之间,继承关系可有可无;
(3)多态的好处可以增加代码的外部调用灵活度,让代码更加通用,兼容性比较强;
(4)多态是调用方法的技巧,不会影响到类的内部设计。
其应用场景主要有以下两类:\\
a. 对象所属的类之间没有继承关系\\
下例中调用同一个函数fly(), 传入不同的参数(对象),可以达成不同的功能。
\begin{lstlisting}
class Duck():                                  #` 鸭子类`
    def fly(self):
        print("`鸭子沿地面飞起`")

class Swan():                                  # `天鹅类`
    def fly(self):
        print("`天鹅在空中翱翔`")

class Plane():                                 # `飞机类`
    def fly(self):
        print("`飞机起飞`")

def fly(obj):                                  # `实现飞的功能函数`
    obj.fly()

duck = Duck()
fly(duck)

swan = Swan()
fly(swan)

plane = Plane()
fly(plane)
 
Output:
`鸭子沿地面飞起`
`天鹅在空中翱翔`
`飞机起飞`
\end{lstlisting}
b. 对象所属的类之间有继承关系。\\
下例中的多态性主要体现向同一个函数,传递不同参数后,可以实现不同功能。
\begin{lstlisting}
class gradapa():
    def _init_(self,money):
        self.money = money
    def p(self):
        print("`this is gradapa`")

class father(gradapa):
    def _init_(self,money,job):
        super()._init_(money)
        self.job = job
    def p(self):
        print("`this is father,重写了父类的方法`")

class mother(gradapa):
    def _init_(self, money, job):
        super()._init_(money)
        self.job = job
    def p(self):
        print("`this is mother,重写了父类的方法`")
        return 100
#`定义一个函数,函数调用类中的p()方法`
def fc(obj):    
    obj.p()

mother1 = mother(1000,"`老师`")
father1 = father(2000,"`工人`")
fc(mother1)
fc(father1)
print(fc(mother1))

Output:
`this is mother,重写了父类的方法`
`this is father,重写了父类的方法`
`100`
\end{lstlisting}

\subsection*{3\;封装(Encapsulation)}
\qquad 封装,是把客观事物封装成抽象的类,并且类可以把自己的数据和方法只让可信的类或者对象操作,对不可信的进行信息隐藏。数据封装就是将某些实例变量隐藏起来,其他程序不能直接访问。\\
\qquad 在Python中,封装实现了对某些方法和属性的“私有化”,将其应用权限限制在某个区域内,外部无法调用。而要对某些特定对象进行私有化时,只需在相应的对象名字前加双下划线。然而,在定义一个类后,如果要对该类中已经私有化的对象进行调用,也可通过一些方法进行实现。比如:在类中专门定义一个函数(方法)用来return该私有化对象,这相当于在类中直接调用,再通过专门定义的函数对私有化对象的值进行return,见下例。\\
\begin{lstlisting}
class Secret:
    __name = 'yoga'                 # `加上双下划线进行私有化`  
    grade = 'high'
A = Secret()
A.grade                             # `可访问到特性`
A.__name                            # `访问不到特性`
\end{lstlisting}
\begin{lstlisting}
class Secret:
    __name = 'yoga'                 # `加上双下划线进行私有化`  
    def get_name(self):      
        return self.__name
A = Secret()
A.get_name()                        # `可访问到类里已私有化的特性`
\end{lstlisting}
\qquad 上述方法不可在外部对私有化对象进行修改,因此可以通过使用函数property来对类中已经私有化的对象进行调用或者修改。有一些面向对象语言支持私有特性,这些特性无法从外部直接访问,需要编写getter和setter方法对这些特性进行操作。Python不需要getter和setter方法,其所有特性都是公开的。对于property函数,我们也可以通过用装饰器函数的方式来进行使用。
\begin{lstlisting}
class Secret:
    def __init__(self):
        self.__name = 'yoda'
        self.work = 'master'
    def get_person(self):
        return self.__name
    def set_person(self,value):
        self.__name = value
    person = property(get_person,set_person)

A = Secret()

print(A.person)
Output:('yoda')

A.person = 'skylaer'
print(A.person)
Output:('skylaer')
\end{lstlisting}
利用装饰器函数。
\begin{lstlisting}
class Secret:
	def __init__(self):
	    self.__name = 'yoda'
	    self.work = 'master'
	@property
	def person(self):
	    return self.__name
	@person.setter
	def person(self,value):
	    self.__name = value
	
A = Secret()
	
print(A.person)
Output:('yoda')
	
A.person = 'skylaer'
print(A.person)
Output:('skylaer')
\end{lstlisting}
\qquad 对于面向对象的类,其封装可以隐藏实现细节,使得代码模块化;继承可以扩展已存在的代码模块(类);它们的目的都是为了代码重用。而多态则是为了实现接口重用,为了类在继承和派生的时候,保证使用任一类的实例的某一属性时的正确调用。


\end{flushleft}
\end{document}